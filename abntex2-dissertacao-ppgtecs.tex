%% Modelo do Programa de Pós Graduação em Tecnologias Sustentáveis
%% baseado no abtex2-modelo-trabalho-academico.tex, v-1.9.6 laurocesar
%% Carlos Guilherme Souza Koehler
%% Copyright 2012-2016 by abnTeX2 group at http://www.abntex.net.br/ 
%%
%% This work may be distributed and/or modified under the
%% conditions of the LaTeX Project Public License, either version 1.3
%% of this license or (at your option) any later version.
%% The latest version of this license is in
%%   http://www.latex-project.org/lppl.txt
%% and version 1.3 or later is part of all distributions of LaTeX
%% version 2005/12/01 or later.
%%
%%
%% This work consists of the files abntex2-modelo-trabalho-academico.tex,
%% abntex2-modelo-include-comandos and abntex2-modelo-references.bib
%%
% ------------------------------------------------------------------------


\documentclass[
	% -- opções da classe memoir --
	12pt,				% tamanho da fonte
	openright,			% capítulos começam em pág ímpar (insere página vazia caso preciso)
	oneside,			% twoside para impressão em recto e verso. Oposto a oneside
	a4paper,			% tamanho do papel. 
	% -- opções da classe abntex2 --
	chapter=TITLE,		% títulos de capítulos convertidos em letras maiúsculas
	%section=TITLE,		% títulos de seções convertidos em letras maiúsculas
	%subsection=TITLE,	% títulos de subseções convertidos em letras maiúsculas
	%subsubsection=TITLE,% títulos de subsubseções convertidos em letras maiúsculas
	% -- opções do pacote babel --
	english,			% idioma adicional para hifenização
	french,				% idioma adicional para hifenização
	spanish,			% idioma adicional para hifenização
	brazil				% o último idioma é o principal do documento
	]{ppgtecs-ifes-abntex2}
	
% ---
% Pacotes básicos 
% ---
	
\usepackage[T1]{fontenc}		% Selecao de codigos de fonte.
\usepackage[utf8]{inputenc}		% Codificacao do documento (conversão automática dos acentos)
\usepackage[scaled]{helvet}
\renewcommand{\familydefault}{\sfdefault}
\usepackage{lastpage}			% Usado pela Ficha catalográfica
\usepackage{indentfirst}		% Indenta o primeiro parágrafo de cada seção.
\usepackage{color}				% Controle das cores
\usepackage{graphicx}			% Inclusão de gráficos
\usepackage{microtype} 			% para melhorias de justificação

		
% ---
% Pacotes adicionais, usados apenas no âmbito do Modelo Canônico do abnteX2
% ---
\usepackage{lipsum}				% para geração de dummy text
% ---

% ---
% Pacotes de citações
% ---
\usepackage[brazilian,hyperpageref]{backref}	 % Paginas com as citações na bibl
\usepackage[alf]{abntex2cite}	% Citações padrão ABNT


% CONFIGURAÇÕES DE PACOTES
% --- 

% ---
% Configurações do pacote backref
% Usado sem a opção hyperpageref de backref
\renewcommand{\backrefpagesname}{Citado na(s) página(s):~}
% Texto padrão antes do número das páginas
\renewcommand{\backref}{}
% Define os textos da citação
\renewcommand*{\backrefalt}[4]{
	\ifcase #1 %
		Nenhuma citação no texto.%
	\or
		Citado na página #2.%
	\else
		Citado #1 vezes nas páginas #2.%
	\fi}%
% ---

% ---

%-------------------------------------------------------------------------
% Comentário adicional do PPGTECS - Informações sobre o ``título'':
%
% Em maiúscula apenas a primeira letra da sentença (do título), exceto 
% nomes próprios, geográficos, institucionais ou Programas ou Projetos ou 
% siglas, os quais podem ter letras em maiúscula também.
%
% O subtítulo do trabalho é opcional.
% Sem ponto final.
%
% Atenção: o título da Dissertação na versão corrigida não pode mudar. 
% Ele deve ser idêntico ao da versão original.
%
%-------------------------------------------------------------------------
\titulo{TÍTULO DO TRABALHO}
%-------------------------------------------------------------------------
% Comentário adicional do PPGTECS - Informações sobre o ``autor'':
%
% Todas as letras em maiúsculas.
% Nome completo.
% Sem ponto final.
%-------------------------------------------------------------------------
\autor{AUTOR DO TRABALHO}
%-------------------------------------------------------------------------
% Comentário adicional do PPGTECS - Informações sobre o ``local'':
%
% Não incluir o ``estado''.
% Sem ponto final.
%-------------------------------------------------------------------------
\local{Vitória}
%-------------------------------------------------------------------------
% Comentário adicional do PPGTECS - Informações sobre a ``data'':
%
% Colocar o ano do depósito (ou seja, o ano da entrega) da respectiva 
% versão, seja ela a versão original (para a defesa) seja ela a versão 
% corrigida (depois da aprovação na defesa). 
%
% Atenção: Se a versão original for depositada no final do ano e a versão 
% corrigida for entregue no ano seguinte, o ano precisa ser atualizado no 
% caso da versão corrigida. 
% Cuidado, pois o ano da ``capa externa'' também precisa ser atualizado 
% nesse caso.
%
% Não incluir o dia, nem o mês.
% Sem ponto final.
%-------------------------------------------------------------------------
\data{2018}
%-------------------------------------------------------------------------
% Comentário adicional do PPGTECS - Informações sobre o ``Orientador'':
%
% Se for uma professora, trocar por ``Profa. Dra.''
% Nome completo.
% Sem ponto final.
%-------------------------------------------------------------------------
\orientador{Prof. Dr. Fulano de Tal}
%-------------------------------------------------------------------------
% Comentário adicional do PPFGTECS- Informações sobre o ``Coorientador'':
%
% Opcional. Incluir apenas se houver co-orientador formal, de acordo com o 
% Regulamento do Programa.
%
% Se for uma professora, trocar por ``Profa. Dra.''
% Nome completo.
% Sem ponto final.
%-------------------------------------------------------------------------
\coorientador{Prof. Dr. Fulano de Tal}
%-------------------------------------------------------------------------
% Comentário adicional do PPGTECS - Informações sobre o ``instituicao'':
%
% Não mexer. Deixar exatamente como está.
%
%-------------------------------------------------------------------------
\instituicao{
	INSTITUTO FEDERAL DO ESPÍRITO SANTO
	\par
	MESTRADO PROFISSIONAL EM TECNOLOGIAS SUSTENTÁVEIS}
\tipotrabalho{Dissertação (Mestrado)}
% O preambulo deve conter o tipo do trabalho, o objetivo, 
% o nome da instituição e a área de concentração 
\preambulo{
	%-------------------------------------------------------------------------
	% Comentário adicional do PPGTECS - Informações sobre o texto ``Versão 
	% original'':
	%
	% Não usar para Qualificação.
	%
	%-------------------------------------------------------------------------
%	Versão original \newline \newline \newline 
	%-------------------------------------------------------------------------
	% Comentário adicional do PPGTECS - Informações sobre o ``texto principal do
	% preambulo'':
	%
	% Para Qualificação, trocar por: Texto de Exame de Qualificação apresentado ao Programa de Pós-Graduação em Tecnologias Sustentáveis do Instituto Federal do Espírito Santo. Como requisito para a obtenção do título de Mestre em Tecnologias Sustentáveis.
	%
	%-------------------------------------------------------------------------
	Dissertação de mestrado apresentado ao Programa de Pós-Graduação em Tecnologias Sustentáveis do Instituto Federal do Espírito Santo. Como requisito para a obtenção do título de Mestre em Tecnologias Sustentáveis. 
	%
	\newline \newline Área de concentração: Interdisciplinar.
	\newline \newline Linha de pesquisa: Otimização de Serviços, Sistemas e Processos.
	\newline \newline Orientador: \imprimirorientador.
	\newline Coorientador: \imprimircoorientador.
%-------------------------------------------------------------------------
% Comentário adicional do PPGTECS - Informações sobre o texto da ``Versão 
% corrigida'':
%
% Não usar para Qualificação.
% Não usar para versão original de Dissertação.
% 
% Substituir ``xx de xxxxxxxxxxxxxxx de xxxx'' pela ``data da defesa''.
%
%-------------------------------------------------------------------------
%\newline \newline \newline Versão corrigida contendo as alterações solicitadas pela comissão julgadora em xx de xxxxxxxxxxxxxxx de xxxx.
}
% ---
% ---
% Configurações de aparência do PDF final

% alterando o aspecto da cor azul
\definecolor{blue}{RGB}{41,5,195}

% informações do PDF
\makeatletter
\hypersetup{
	%pagebackref=true,
	pdftitle={\@title}, 
	pdfauthor={\@author},
	pdfsubject={\imprimirpreambulo},
	pdfcreator={LaTeX with abnTeX2},
	pdfkeywords={abnt}{latex}{abntex}{abntex2}{trabalho acadêmico}, 
	colorlinks=true,       		% false: boxed links; true: colored links
	linkcolor=black,          	% color of internal links
	citecolor=black,        		% color of links to bibliography
	filecolor=black,      		% color of file links
	urlcolor=black,
	bookmarksdepth=4
}
\makeatother
% --- 

% --- 
% Espaçamentos entre linhas e parágrafos 
% --- 

% O tamanho do parágrafo é dado por:
\setlength{\parindent}{1.3cm}

% Controle do espaçamento entre um parágrafo e outro:
\setlength{\parskip}{0.2cm}  % tente também \onelineskip

% ---
% compila o indice
% ---
\makeindex
% ---

% ----
% Início do documento
% ----
\begin{document}

% Seleciona o idioma do documento (conforme pacotes do babel)
%\selectlanguage{english}
\selectlanguage{brazil}

% Retira espaço extra obsoleto entre as frases.
\frenchspacing 

% ----------------------------------------------------------
% ELEMENTOS PRÉ-TEXTUAIS
% ----------------------------------------------------------
% \pretextual

% ---
% Capa
% ---
\imprimircapa
% ---
% ---
% Folha de rosto
% (o * indica que haverá a ficha bibliográfica)
% ---
\imprimirfolhaderosto*
% ---
% ---

% ---
% Insere a ficha bibliográfica, para inserir a ficha em pdf comentar a primeira
% linha e tirar o comentário da segunda.
% ---
% Inserir a ficha bibliografica
% ---
% Isto é um exemplo de Ficha Catalográfica, ou ``Dados internacionais de
% catalogação-na-publicação''. Você pode utilizar este modelo como referência. 
% Porém, provavelmente a biblioteca da sua universidade lhe fornecerá um PDF
% com a ficha catalográfica definitiva após a defesa do trabalho. Quando estiver
% com o documento, salve-o como PDF no diretório do seu projeto e substitua todo
% o conteúdo de implementação deste arquivo pelo comando abaixo:
%
% \begin{fichacatalografica}
%     \includepdf{fig_ficha_catalografica.pdf}
% \end{fichacatalografica}

\begin{fichacatalografica}
	\sffamily
	\vspace*{\fill}					% Posição vertical
	\begin{center}					% Minipage Centralizado
	\fbox{\begin{minipage}[c][8cm]{13.5cm}		% Largura
	\small
	\imprimirautor
	%Sobrenome, Nome do autor
	
	\hspace{0.5cm} \imprimirtitulo  / \imprimirautor. --
	\imprimirlocal, \imprimirdata-
	
	\hspace{0.5cm} \pageref{LastPage} p. : il. (algumas color.) ; 30 cm.\\
	
	\hspace{0.5cm} \imprimirorientadorRotulo~\imprimirorientador\\
	
	\hspace{0.5cm}
	\parbox[t]{\textwidth}{\imprimirtipotrabalho~--~\imprimirinstituicao,
	\imprimirdata.}\\
	
	\hspace{0.5cm}
		1. Palavra-chave1.
		2. Palavra-chave2.
		2. Palavra-chave3.
		I. Orientador.
		II. Universidade xxx.
		III. Faculdade de xxx.
		IV. Título 			
	\end{minipage}}
	\end{center}
\end{fichacatalografica}
% ---
% \includepdf{ficha-bibliografica.pdf}
% ---

% ---
% Insere a errata
% ---
% ---
% Inserir errata
% ---
\begin{errata}
Elemento opcional da \citeonline[4.2.1.2]{NBR14724:2011}. Exemplo:

\vspace{\onelineskip}

FERRIGNO, C. R. A. \textbf{Tratamento de neoplasias ósseas apendiculares com
reimplantação de enxerto ósseo autólogo autoclavado associado ao plasma
rico em plaquetas}: estudo crítico na cirurgia de preservação de membro em
cães. 2011. 128 f. Tese (Livre-Docência) - Faculdade de Medicina Veterinária e
Zootecnia, Universidade de São Paulo, São Paulo, 2011.

\begin{table}[htb]
\center
\footnotesize
\begin{tabular}{|p{1.4cm}|p{1cm}|p{3cm}|p{3cm}|}
  \hline
   \textbf{Folha} & \textbf{Linha}  & \textbf{Onde se lê}  & \textbf{Leia-se}  \\
    \hline
    1 & 10 & auto-conclavo & autoconclavo\\
   \hline
\end{tabular}
\end{table}

\end{errata}
% ---

% \includepdf{errata.pdf}
% ---

% ---
% Insere a folha de aprovação, para inserir a folha de aprovação em pdf 
% comentar a primeira linha e tirar o comentário da segunda.
% ---
% ---
% Inserir folha de aprovação
% ---

% Isto é um exemplo de Folha de aprovação, elemento obrigatório da NBR
% 14724/2011 (seção 4.2.1.3). Você pode utilizar este modelo até a aprovação
% do trabalho. Após isso, substitua todo o conteúdo deste arquivo por uma
% imagem da página assinada pela banca com o comando abaixo:
%
% \includepdf{folhadeaprovacao_final.pdf}
%
\begin{folhadeaprovacao}

  \begin{center}
    {\ABNTEXchapterfont\imprimirautor}

    \vspace*{\fill}\vspace*{\fill}
    \begin{center}
      Referência do artigo submetido
    \end{center}
    \vspace*{\fill}
    
    \hspace{.45\textwidth}
    \begin{minipage}{.5\textwidth}
        Produto final apresentado ao Programa de Pós-Graduação em Tecnologias Sustentáveis do Instituto Federal do Espírito Santo. Como requisito para a obtenção do título de Mestre em Tecnologias Sustentáveis.
    \end{minipage}%
    \vspace*{\fill}
   \end{center}
        
   Aprovado em  XX de mês de XXXX:

   \assinatura{\textbf{\imprimirorientador} \\ Instituto Federal do Espírito Santo \\ Orientador} 
   \assinatura{\textbf{Professor 1} \\ Instituto Federal do Espírito Santo \\Membro interno}
   \assinatura{\textbf{Professor 2} \\ Instituto Federal do Espírito Santo \\Membro interno}
   \assinatura{\textbf{Professor 3} \\ Universidade do Brasil \\Membro externo}
   %\assinatura{\textbf{Professor} \\ Convidado 4}
        
\end{folhadeaprovacao}
% ---
% \includepdf{folha-aprovacao.pdf}
% ---

% ---
% Insere a dedicatoria
% ---
% ---
% Dedicatória
% ---
\begin{dedicatoria}
   \vspace*{\fill}
   \centering
   \noindent
   \textit{ Este trabalho é dedicado às crianças adultas que,\\
   quando pequenas, sonharam em se tornar cientistas.} \vspace*{\fill}
\end{dedicatoria}
% ---
% ---

% ---
% Insere os agradecimentos
% ---
% ---
% Agradecimentos
% ---
\begin{agradecimentos}
Os agradecimentos principais são direcionados aos docentes do Programa de Pós Graduação em Tecnologias Sustentáveis (PPGTECS) do Instituto Federal do Espírito Santo(IFES) por se dedicarem a esse projeto proporcionando uma qualificação técnica de excelência aos discentes.

Agradecimentos especiais são direcionados aos inúmeros profissionais do IFES que estão dando suporte ao PPGTECS.

\end{agradecimentos}
% ---
% ---

% ---
% Insere o epígrafe
% ---

% ---
% Epígrafe
% ---
\begin{epigrafe}
    \vspace*{\fill}
	\begin{flushright}
		\textit{“Nobody ever figures out what life is all\\
			about, and it doesn’t matter. Explore the\\
			world. Nearly everything is really\\
			interesting if you go deeply enough.”\\		
		(Richard P. Feynman)}
	\end{flushright}
\end{epigrafe}
% ---
% ---

% ---
% Insere os resumos
% ---
% resumo em português
\setlength{\absparsep}{18pt} % ajusta o espaçamento dos parágrafos do resumo
\begin{resumo}
 Segundo a \citeonline[3.1-3.2]{NBR6028:2003}, o resumo deve ressaltar o objetivo, o método, os resultados e as conclusões do documento. O resumo deve dar uma visão concisa e objetiva do conteúdo da pesquisa e ser redigido pelo próprio autor. A ordem e a extensão destes itens dependem do tipo de resumo (informativo ou indicativo) e do tratamento que cada item recebe no documento original. Deve-se fazer uso da terceira pessoa no singular, verbo na voz ativa, ser redigido em parágrafo único, espaço de entrelinhas um e meio (1,5) e conter no mínimo 150 e no máximo 500 palavras. O resumo deve ser redigido na mesma língua do texto e estar localizado em folha separada.
 
As palavras-chave ou descritores devem constar logo abaixo do texto do resumo, no máximo 5 palavras,separadas entre si por ponto e finalizadas também por ponto. 

 \textbf{Palavras-chave}: latex. abntex. editoração de texto.
\end{resumo}

% resumo em inglês
\begin{resumo}[Abstract]
 \begin{otherlanguage*}{english}
   This is the english abstract.

   \vspace{\onelineskip}
 
   \noindent 
   \textbf{Keywords}: latex. abntex. text editoration.
 \end{otherlanguage*}
\end{resumo}

% resumo em francês 
\begin{resumo}[Résumé]
 \begin{otherlanguage*}{french}
    Il s'agit d'un résumé en français.
 
   \textbf{Mots-clés}: latex. abntex. publication de textes.
 \end{otherlanguage*}
\end{resumo}

% resumo em espanhol
\begin{resumo}[Resumen]
 \begin{otherlanguage*}{spanish}
   Este es el resumen en español.
  
   \textbf{Palabras clave}: latex. abntex. publicación de textos.
 \end{otherlanguage*}
\end{resumo}
% ---
% ---

% ---
% Insere as listas de ilustrações, tabelas, abreviaturas e símbolos.
% ---
% ---
% inserir lista de ilustrações
% ---
\pdfbookmark[0]{\listfigurename}{lof}
\listoffigures*
\cleardoublepage
% ---

% ---
% inserir lista de tabelas
% ---
\pdfbookmark[0]{\listtablename}{lot}
\listoftables*
\cleardoublepage
% ---

% ---
% inserir lista de abreviaturas e siglas
% ---
\begin{siglas}
  \item[ABNT] Associação Brasileira de Normas Técnicas
  \item[abnTeX] ABsurdas Normas para TeX
\end{siglas}
% ---

% ---
% inserir lista de símbolos
% ---
\begin{simbolos}
  \item[$ \Gamma $] Letra grega Gama
  \item[$ \Lambda $] Lambda
  \item[$ \zeta $] Letra grega minúscula zeta
  \item[$ \in $] Pertence
\end{simbolos}
% ---
% ---

% ---
% inserir o sumario
% ---
\pdfbookmark[0]{\contentsname}{toc}
\tableofcontents*
\cleardoublepage
% ---



% ----------------------------------------------------------
% ELEMENTOS TEXTUAIS
% ----------------------------------------------------------
\textual

% ---
% Insere a introdução.
% ---
% ----------------------------------------------------------
% Introdução (exemplo de capítulo sem numeração, mas presente no Sumário)
% ----------------------------------------------------------
\chapter{Introdução}
\setlength\afterchapskip{\lineskip}
% ----------------------------------------------------------

Este documento e seu código-fonte são exemplos de referência de uso da classe
\textsf{ppgtecs-ifes-abntex2} e do pacote \textsf{abntex2cite}. O documento exemplifica a elaboração da dissertação produzida conforme a ABNT NBR 14724:2011 \emph{Informação e documentação - Trabalhos acadêmicos - Apresentação} e o documento \emph{Normas para apresentação de trabalhos acadêmicos e científicos: documento impresso e/ou digital. 8 ed. rev. e ampl. – Vitória : Ifes, 2017}.

Encorajo que sejam realizadas customizações desse exemplo para o uso de todos os discentes do PPGTECS. 

Este documento deve ser utilizado como complemento dos manuais do \abnTeX\ 
\cite{abntex2classe,abntex2cite,abntex2cite-alf} e da classe \textsf{memoir}
\cite{memoir}. 

Carlos Guilherme Souza Koehler
% ---


% ---
% Capitulo de revisão de literatura
% ---
\chapter{Lorem ipsum dolor sit amet}
\setlength\afterchapskip{\lineskip}
% ---

% ---
\section{ALIQUAM VESTIBULUM FRINGILLA LOREM}
% ---

\lipsum[1]

\lipsum[2-3]


% ---
% primeiro capitulo de Resultados
% ---
\chapter{Lectus lobortis condimentum}
\setlength\afterchapskip{\lineskip}
% ---

% ---
\section{VESTIBULUM ANTE IPSUM PRIMIS IN FAUCIBUS ORCI LUCTUS ET ULTRICES POSUERE CUBILIA CURAE}
% ---

\lipsum[21-22]

% ---
% segundo capitulo de Resultados
% ---
\chapter{Nam sed tellus sit amet lectus urna ullamcorper tristique interdum
elementum}
\setlength\afterchapskip{\lineskip}
% ---

% ---
\section{PELLENTESQUE SIT AMET PEDE AC SEM ELEIFEND COSECTETUER}
% ---

\lipsum[24]

% ----------------------------------------------------------
% Finaliza a parte no bookmark do PDF
% para que se inicie o bookmark na raiz
% e adiciona espaço de parte no Sumário
% ----------------------------------------------------------
\phantompart

% ---
% Conclusão
% ---
\chapter{Conclusão}
\setlength\afterchapskip{\lineskip}
% ---

\lipsum[31-33]

% ----------------------------------------------------------
% ELEMENTOS PÓS-TEXTUAIS
% ----------------------------------------------------------
\postextual
% ----------------------------------------------------------

% ----------------------------------------------------------
% Referências bibliográficas
% ----------------------------------------------------------
\bibliography{referencias}

% ----------------------------------------------------------
% Glossário
% ----------------------------------------------------------
%
% Consulte o manual da classe abntex2 para orientações sobre o glossário.
%
%\glossary


%---------------------------------------------------------------------
% INDICE REMISSIVO
%---------------------------------------------------------------------
\phantompart
\printindex
%---------------------------------------------------------------------

\end{document}
